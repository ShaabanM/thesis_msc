The most revolutionary discovery in cosmology since 
Hubble observed that the Universe is expanding is that 
this expansion is accelerating. A revelation that was 
awarded the 2011 Nobel Prize for its profound 
implications. \cite{nobel}. An accelerating
universe implies that either our understanding of gravity is flawed 
or that a mysterious negative pressure known as Dark Energy is driving the 
expansion \cite{peebles}.
This Dark Energy accounts for most (over 68\%) of the energy density in the observable universe, 
however its origin and physics are presently unknown \cite{planck}. 
As a result, the nature of Dark Energy is considered one of the 
greatest mysteries of modern science \cite{pathfinder}.  
\par
One of the most powerful ways to probe Dark Energy, as well as modified theories of gravity, is a technique known as weak gravitational lensing, or weak lensing for short \cite{hoekstra,rachel_2018}. Gravitational lensing is the phenomenon of light ray deflection by intervening mass. When the deflection is sufficiently weak, this phenomenon manifests in images of galaxies as a shearing effect due to the differential deflection of neighboring light rays \cite{general_2013,hoekstra}. This shearing induces a subtle (sub 1\%) change in the ellipticity of the images. Although such a change is negligible in comparison to the 30\% dispersion in intrinsic galaxy ellipticities, it can be statistically measured by using the coherence of the lensing shear over the sky \cite{general_2013}. The reason weak lensing is considered very powerful is because it provides a direct measurement of the matter distribution in the universe as a function of redshift independent of any cosmological assumptions. Thus, allowing us to directly probe the growth of cosmic structure with time \cite{hoekstra}. 
\par
In this report I present a general overview of weak lensing in the context of cosmology. I begin by presenting the theory behind the Cosmology we would like to probe, as well as the general theoretical framework on which weak lensing operates. I then present the weak lensing measurment procedure and how cosmological data can be extracted from such measurements. Finally, I end this report by presenting some results from an active experiment and a short discussion on potentially concerning systematics. 